\section{Formula��o PLI para o Problema}
\label{sec:form_pcmcrc}

A formula��o do \pcmcrc{} � a mesma formula��o do \pcmc{} adicionando a restri��o de custo $c(P^{*}) \le C$.
Ent�o temos o seguinte modelo para o \mpcmcrc{}: \\

Fun��o Objetivo:

\begin{equation}
\label{eq:fo_pli}
z = \min \sum_{(i, j) \in E} d_{i, j} \cdot x_{i, j}
\end{equation} \\

Sujeito a:

\begin{align}
\label{eq:restr_pli_pcmc1}
&\sum_{j \in V} x_{s, j} = 1 \\
\label{eq:restr_pli_pcmc2}
&\sum_{i \in V} x_{i, t} = 1 \\
\label{eq:restr_pli_pcmc3}
&\sum_{j \in V} x_{i, j} - \sum_{j \in V} x_{j, i} = 0, \qquad \forall i \in V\backslash\{s, t\}
\end{align}

\begin{equation}
\label{eq:restr_pli_knap}
\sum_{(i, j) \in E} c_{i, j} . x_{i, j} \le C
\end{equation}

\begin{equation}
\label{eq:restr_pli_x}
x_{i, j} \in \mathbb{B}, \qquad \forall (i, j) \in E
\end{equation} \\

Onde: 

\begin{itemize}
\item{$d_{i, j}$: Dist�ncia do v�rtice $i$ para o v�rtice $j$.}
\item{$c_{i, j}$: Custo do v�rtice $i$ para o v�rtice $j$.}
\item{$x_{i, j}$: 1 se a aresta ($i$, $j$) pertence ao caminho e 0 caso contr�rio.}
\item{$s$: V�rtice origem.}
\item{$t$: V�rtice destino.}
\end{itemize}

Observe que as restri��es (\ref{eq:restr_pli_pcmc1}), (\ref{eq:restr_pli_pcmc2}) e (\ref{eq:restr_pli_pcmc3}) 
s�o o conjunto de restri��es do problema cl�ssico do \cmc{}, onde:

\begin{itemize}
\item{Restri��o (\ref{eq:restr_pli_pcmc1}): S� � permitida uma aresta saindo do v�rtice origem $s$.}
\item{Restri��o (\ref{eq:restr_pli_pcmc2}): S� � permitida uma aresta entrando no v�rtice destino $t$.}
\item{Restri��o (\ref{eq:restr_pli_pcmc3}): Um v�rtice $i \in V\backslash\{s, t\}$ no caminho deve possuir apenas 
uma aresta entrando e uma aresta saindo.}
\end{itemize}

A restri��o (\ref{eq:restr_pli_knap}) � a restri��o do tipo \knap{} que representa a restri��o de custo.
