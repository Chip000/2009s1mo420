\section{Limitante Primal}
\label{sec:limprim}

O Limitante Primal (Limitante Superior) para o \pcmcrc{} � encontrado fazendo os seguintes passos:

\begin{enumerate}
\item{$l \leftarrow 1$.}
\item{Seja $d[i][j]$ o valor da dist�ncia da aresta $(i, j)$ e $c[i][j]$ o valor do custo da aresta $(i, j)$.}
\item{Enquanto $l \ge 0$, Executar o shortest\_path()~(\ref{alg:shortest_path}), 
usando como dist�ncia $l \cdot d[i][j] + (1 - l) \cdot c[i][j]$ para todas as arestas $(i, j) \in E$}
\item{Verificar se a restri��o de custo � satisfeita. Se for satisfeita v� para o passo 6.}
\item{Sen�o $l \leftarrow l - \xi$ onde $ 0 < \xi < 1$ e volte para passo 3.}
\item{Calcular o valor da solu��o encontrada.}
\end{enumerate}

No melhor dos casos o limitante primal vai ser o valor da solu��o �tima do \mpcmcrc{}. Quando $l = 0$ o problema
vai minimizar o caminho usando os custos como dist�ncia, se este caminho respeita a restri��o de custo,
esta solu��o ir� ser vi�vel.

